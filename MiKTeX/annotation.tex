\sectioncentered*{Аннотация}
\thispagestyle{empty}

\begin{center}
  \begin{minipage}{0.82\textwidth}
    на дипломный проект <<Алгоритмы построения вероятностных сетей>> студента УО <<Белорусский государственный университет информатики и радиоэлектроники>> Ярошевича~Ю.\,А.
  \end{minipage}
\end{center}

\emph{Ключевые слова}: вероятностные модели; байесовы сети; вывод структуры сети по данным; принцип минимальной длинны описания; оценка апостериорной вероятности.

\vspace{4\parsep}

Дипломный проект выполнен на 6 листах формата А1 с пояснительной запиской на~\pageref*{LastPage} страницах, без приложений справочного или информационного характера. 
Пояснительная записка включает \total{section}~глав, \totfig{}~рисунков, \tottab{}~таблиц, \toteq{}~формулы, \totref{}~литературный источник.

Темой дипломного проекта является построение структуры вероятностной сети по данным без участия эксперта.
Целю проекта является разработка библиотеки кода предназначенной для работы с вероятностными сетями и автоматического вывода структуры сети по данным.

Во введении производится ознакомление с проблемой, решаемой в дипломном проекте.

В первой главе производится обзор предметной области проблемы решаемой в данном дипломном проекте.
Приводятся необходимые теоретические сведения, а также производится обзор существующих разработок.

Во второй главе производится краткий обзор технологий, использованных для реализации ПО в рамках дипломного проекта.

В третьей главе производится обзор реализованного ПО.
Описываются его составные части и особенности.
Приводятся результаты практических испытаний и производится сравнение с существующим ПО.

В четвертой главе производится оценка пожарной безопасности предприятия, на котором разрабатывался данный дипломный проект.

В пятой главе производится технико"=экономическое обоснование разработки и производится оценка прогнозируемой прибыли у разработчика от реализации проекта.

В заключении подводятся итоги и делаются выводы по дипломному проекту, а также описывается дальнейший план развития проекта.

\clearpage