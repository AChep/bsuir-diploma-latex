% !TeX spellcheck = russian-aot-ieyo
% Зачем: Определяет класс документа (То, как будет выглядеть документ)
% Примечание: параметр draft помечает строки, вышедшие за границы страницы, прямоугольником, в фильной версии его нужно удалить.
\documentclass[a4paper,14pt,russian,oneside,draft]{extreport}

% Зачем: Установка кодировки исходных файлов.
\usepackage[utf8]{inputenc}

% Зачем: Делает результирующий PDF "searchable and copyable".
\usepackage{cmap}

% Зачем: Выбор внутренней TeX кодировки.
\usepackage[T2A]{fontenc}

% Зачем: Чтобы можно было использовать русские буквы в формулах, но в случае использования предупреждать об этом.
\usepackage[warn]{mathtext}

% Зачем: Учет особенностей различных языков.
\usepackage[russian]{babel}

% Зачем: Улучшает отображение русских шрифтов.
% Примечание: Требует шаманства при установке, инструкция http://plumbum-blog.blogspot.com/2010/06/miktex-28-pscyr-04d.html
\usepackage{pscyr}


% Зачем: Добавляет поддержу дополнительных размеров текста 8pt, 9pt, 10pt, 11pt, 12pt, 14pt, 17pt, and 20pt.
% Почему: Пункт 2.1.1 Требований по оформлению пояснительной записки.
\usepackage{extsizes}


% Зачем: Длинна, пимерно соответвующая 5 символам
% Почему: Требования содержат странное требование про отсупы в 5 символов (для немоноширинного шрифта :| )
\newlength{\fivecharsapprox}
\setlength{\fivecharsapprox}{6ex}


% Зачем: Добавляет отступы для абзацев.
% Почему: Пункт 2.1.3 Требований по оформлению пояснительной записки.
\usepackage{indentfirst}
\setlength{\parindent}{\fivecharsapprox} % Примерно соответсвует 5 символам.


% Зачем: Настраивает отступы от границ страницы.
% Почему: Пункт 2.1.2 Требований по оформлению пояснительной записки.
\usepackage[left=3cm,top=2.0cm,right=1.5cm,bottom=2.7cm]{geometry}


% Зачем: Настраивает межстрочный интервал, для размещения 40 +/- 3 строки текста на странице.
% Почему: Пункт 2.1.1 Требований по оформлению пояснительной записки.
\usepackage[nodisplayskipstretch]{setspace} 
\setstretch{1.1}


% Зачем: Выбор шрифта по-умолчанию. 
% Почему: Пункт 2.1.1 Требований по оформлению пояснительной записки.
% Примечание: В требованиях не указан, какой именно шрифт использовать. По традиции используем TNR.
\renewcommand{\rmdefault}{ftm} % Times New Roman


% Зачем: Отключает использование изменяемых межсловных пробелов.
% Почему: Так не принято делать в текстах на русском языке.
\frenchspacing


% Зачем: Счетчик страниц начинается с 3. Пропускается титульный лист, лист с рефератом и лист технического задания.
% Почему: Пункт 2.2.8 Требований по оформлению пояснительной записки.
\setcounter{page}{3}


% Зачем: Сброс счетчика сносок для каждой страницы
% Примечание: в "Требованиях по оформлению пояснительной записки" не указано, как нужно делать, но в других БГУИРовских докуметах рекомендуется нумерация отдельная для каждой страницы
\usepackage{perpage}
\MakePerPage{footnote}


% Зачем: Добавляет скобку 1) к номеру сноски
% Почему: Пункты 2.9.2 и 2.9.1 Требований по оформлению пояснительной записки.
\makeatletter 
\def\@makefnmark{\hbox{\@textsuperscript{\normalfont\@thefnmark)}}}
\makeatother


% Зачем: Расположение сносок внизу страницы
% Почему: Пункт 2.9.2 Требований по оформлению пояснительной записки.
\usepackage[bottom]{footmisc}


% Зачем: Переопределяем стандартную нумерацию, т.к. в отчете будут только section и т.д. в терминологии TeX
\makeatletter
\renewcommand{\thesection}{\arabic{section}}
\makeatother


% Зачем: Пункты (в терминологии требований) в терминологии TeX subsubsection должны нумероваться
% Почему: Пункт 2.2.3 Требований по оформлению пояснительной записки.
\setcounter{secnumdepth}{3}


% Зачем: Настраивает отступ между таблицей с содержанимем и словом СОДЕРЖАНИЕ
% Почему: Пункт 2.2.7 Требований по оформлению пояснительной записки.
\usepackage{tocloft}
\setlength{\cftbeforetoctitleskip}{-1em}
\setlength{\cftaftertoctitleskip}{1em}


% Зачем: Определяет отступы слева для записей в таблице содержания.
% Почему: Пункт 2.2.7 Требований по оформлению пояснительной записки.
\makeatletter
\renewcommand{\l@section}{\@dottedtocline{1}{0.5em}{1.2em}}
\renewcommand{\l@subsection}{\@dottedtocline{2}{1.7em}{2.0em}}
\makeatother


% Зачем: Работа с колонтитулами
\usepackage{fancyhdr} % пакет для установки колонтитулов
\pagestyle{fancy} % смена стиля оформления страниц


% Зачем: Нумерация страниц располагается справа снизу страницы
% Почему: Пункт 2.2.8 Требований по оформлению пояснительной записки.
\fancyhf{} % очистка текущих значений
\fancyfoot[R]{\thepage} % установка верхнего колонтитула
\renewcommand{\footrulewidth}{0pt} % убрать разделительную линию внизу страницы
\renewcommand{\headrulewidth}{0pt} % убрать разделительную линию вверху страницы
\fancypagestyle{plain}{ 
    \fancyhf{}
    \rfoot{\thepage}}


% Зачем: Задает стиль заголовков раздела жирным шрифтом, прописными буквами, без точки в конце
% Почему: Пункты 2.1.1, 2.2.5, 2.2.6 и ПРИЛОЖЕНИЕ Л Требований по оформлению пояснительной записки.
\makeatletter
\renewcommand\section{%
  \@startsection {section}{1}%
    {\fivecharsapprox}%
    {-1em \@plus -1ex \@minus -.2ex}%
    {1em \@plus .2ex}%
    {\newpage\raggedright\hyphenpenalty=10000\normalfont\large\bfseries\MakeUppercase}}
\makeatother


% Зачем: Задает стиль заголовков подразделов
% Почему: Пункты 2.1.1, 2.2.5 и ПРИЛОЖЕНИЕ Л Требований по оформлению пояснительной записки.
\makeatletter
\renewcommand\subsection{%
  \@startsection{subsection}{2}%
    {\fivecharsapprox}%
    {-1em \@plus -1ex \@minus -.2ex}%
    {1em \@plus .2ex}%
    {\raggedright\hyphenpenalty=10000\normalfont\normalsize\bfseries}}
\makeatother


% Зачем: Задает стиль заголовков пунктов
% Почему: Пункты 2.1.1, 2.2.5 и ПРИЛОЖЕНИЕ Л Требований по оформлению пояснительной записки.
\makeatletter
\renewcommand\subsubsection{
  \@startsection{subsubsection}{3}%
    {\fivecharsapprox}%
    {-1em \@plus -1ex \@minus -.2ex}%
    {\z@}%
    {\raggedright\hyphenpenalty=10000\normalfont\normalsize\bfseries}}
\makeatother




\begin{document}
% Данный файл предназначен для включения внутри окружения doucment с помощью команды \input


% Зачем: Содержание пишется полужирным шрифтом, по центру всеми заглавными буквами
% Почему: Пункт 2.2.7 Требований по оформлению пояснительной записки.
\renewcommand \contentsname {\centerline{\mdseries\large{СОДЕРЖАНИЕ}}}


% Зачем: Изменение надписи для списка литературы
% Почему: Пункт 2.8.1 Требований по оформлению пояснительной записки.
\renewcommand\bibname{\bf \normalsize \centerline{СПИСОК ИСПОЛЬЗОВАННЫХ ИСТОЧНИКОВ}}


\tableofcontents
\pagebreak


%\section{Пояснительную  записку  выполняют  рукописным  способом  или  с применением печатающих и графических устройств вывода ЭВМ}
Пояснительную  записку  выполняют  рукописным  способом  или  с применением печатающих и графических устройств вывода ЭВМ. 

При рукописном способе используют шариковую ручку с пастой черного или синего, или фиолетового цвета. Высота букв и цифр должна быть не менее 3,5 мм. 

При применении текстовых редакторов ЭВМ печать производится шрифтом 13\,--\,14  пунктов  с  межстрочным  интервалом,  позволяющим  разместить  
40\,\( \pm \)\,3 строки на странице. 

Номера  разделов,  подразделов,  пунктов и подпунктов следует выделять полужирным  шрифтом.  Заголовки  разделов  допускается  оформлять  полужирным шрифтом размером 14\,--\,16 пунктов, а заголовки подразделов полужирным шрифтом размером 14 пунктов. 

Для  акцентирования  внимания  на  определенных  терминах  допускается применять шрифты разной гарнитуры. 

Абзацы в тексте начинают отступом, равным 15\,--\,17 мм при выполнении  записи  рукописным  способом  или  пяти  знакам  при  применении  печатающего устройства вывода ЭВМ.

Все  части  пояснительной  записки  необходимо  излагать  только  на одном языке --- на русском или белорусском, или на одном из иностранных языков, например английском или немецком.

Описки и графические неточности, обнаруженные в тексте пояснительной записки, выполненной рукописным способом, допускается исправлять подчисткой, закрашиванием белой краской и нанесением на том же месте исправленного текста. Помарки и следы не полностью удаленного прежнего текста не допускаются.

Пояснительная записка\footnote{Ага, должна, от гопников удобнее отбиваться} должна быть оформлена в жестком переплете (в специальной папке для дипломных проектов или работ).

Текст пояснительной записки разделяют на логически сопряженные части --- разделы, а при необходимости и подразделы. Как разделы, так и подразделы могут состоять из одного или нескольких пунктов.

\doublehyphendemerits=100000
Иногда внутри подраздела необходимо выделить более мелкие смысловые подразделения – пункты, например: характеристики устройств и функциональных элементов технической системы; обоснование этапов планируемого эксперимента, характеристики аппаратов и приборов, необходимых для испытаний; показатели качества технической системы в различных режимах ее работы и т.\,д. В подобных случаях пункты нумеруют в пределах подраздела. Цифровой индекс пункта должен состоять из номеров раздела, подраздела и пункта, разделенных точками, и записан с абзацного отступа. 

Пункты при необходимости могут быть разбиты на подпункты, которые нумеруются в пределах каждого пункта. 

\noindent\newlength{\ldn}%
\ldn=1em 1\,em равен \the\ldn\\
\ldn=1ex 1\,ex равен \the\ldn

\newlength{\ldm}\settowidth{\ldm}{M}\the\ldm

{ 
\fboxrule=1ex
\fbox { Hello }
}
\fbox { world }

\noindent Hello World!\\
\hspace{10mm}Hello World!\\
\hspace*{10mm}Hello World!\\
Hello\hspace*{-2mm}World!\\


Hello World!\\
Hello\hfill World!\\
A\hfill\hfill{}B\hfill{}C\\
Hello\dotfill{}World!\\
Hello\hrulefill{}World!\\

Hello\rule{1cm}{.3pt}World!\\
Hello \rule{10pt}{10pt} World!\\
Hello \rule[2pt]{1cm}{3pt} World!\\
Hello \rule[-2pt]{1cm}{3pt} World!\\
Примечание\footnote{Сноска 1}

\the\doublehyphendemerits\\
\the\finalhyphendemerits\\
\the\righthyphenmin\\
\the\lefthyphenmin\\
\the\clubpenalty\\
\the\widowpenalty\\
\the\pretolerance\\
\the\tolerance\\
Примечание\footnote{Сноска 2}
%\vfill


% Примеры разделов
\section*{Введение}
\addcontentsline{toc}{section}{Введение} \pagebreak
\section{Анализ нескорректированной системы управления} \pagebreak
\subsection{Анализ исходных данных} \pagebreak 
\subsection{Статические и динамические характеристики элементов системы} \pagebreak 
\subsection{Структурная схема нескорректированной системы} \pagebreak 
\subsection{Определение желаемого коэффициента усиления разомкнутой системы} \pagebreak 
\subsection{Анализ устойчивости} \pagebreak 
\subsection{Выводы} \pagebreak 
\subsubsection{Пункт подраздела, не должен появится в содержании} \pagebreak
\section{Синтез корректирующих устройств} \pagebreak
\section*{Приложение А (информационное)  Пример заполнения титульного листа} 
\addcontentsline{toc}{section}{Приложение А (информационное)  Пример заполнения титульного листа} \pagebreak


\begin{thebibliography}{99}
\end{thebibliography}

\end{document}