% !TeX spellcheck = russian-aot-ieyo
\documentclass[a4paper,14pt,russian,oneside]{extreport}

% Зачем: Установка кодировки исходных файлов
\usepackage[utf8]{inputenc}

% Зачем: Выбор внутренней TeX кодировки
\usepackage[T2A]{fontenc}

% Зачем: Чтобы можно было использовать русские буквы в формулах, но в случае использования предупреждать об этом
\usepackage[warn]{mathtext}

% Зачем: Учет особенностей различных языков
\usepackage[russian]{babel}

% Зачем: Делает результирующий PDF "searchable and copyable"
\usepackage{cmap}

% Зачем: Улучшает отображение русских шрифтов.
% Примечание: Требует шаманства при установке, инструкция http://plumbum-blog.blogspot.com/2010/06/miktex-28-pscyr-04d.html
\usepackage{pscyr}


% Зачем: Добавляет поддержу дополнительных размеров текста 8pt, 9pt, 10pt, 11pt, 12pt, 14pt, 17pt, and 20pt.
% Требования: Пункт 2.1.1 Требований по оформлению пояснительной записки
\usepackage{extsizes}


% Зачем: Добавляет отступы для абзацев
% Требования: Пункт 2.1.3 Требований по оформлению пояснительной записки
\usepackage{indentfirst}
\parindent=2.8em % Примерно соответсвует 5 символам


% Зачем: Настраивает отступы от границ страницы
% Требования: Пункт 2.1.2
\usepackage[left=3cm,top=2.0cm,right=1.5cm,bottom=2.7cm]{geometry}

