
\section{Требования к пояснительной записке}

\subsection{Общие положения}

\subsubsection{} 
Пояснительную  записку  выполняют  рукописным  способом  или  с применением печатающих и графических устройств вывода ЭВМ. 

При рукописном способе используют шариковую ручку с пастой черного или синего, или фиолетового цвета. Высота букв и цифр должна быть не менее 3,5 мм. 

При применении текстовых редакторов ЭВМ печать производится шрифтом 13\,--\,14  пунктов  с  межстрочным  интервалом,  позволяющим  разместить  
40\,\( \pm \)\,3 строки на странице. 

Номера  разделов,  подразделов,  пунктов и подпунктов следует выделять полужирным  шрифтом.  Заголовки  разделов  допускается  оформлять  полужирным шрифтом размером 14\,--\,16 пунктов, а заголовки подразделов полужирным шрифтом размером 14 пунктов. 

Для  акцентирования  внимания  на  определенных  терминах  допускается применять шрифты разной гарнитуры. 

\subsubsection{}
Текст располагают на одной стороне листа формата А4 с соблюдением размеров полей и интервалов, указанных в приложении Л.

\subsubsection{}
Абзацы в тексте начинают отступом, равным 15\,--\,17 мм при выполнении  записи  рукописным  способом  или  пяти  знакам  при  применении  печатающего устройства вывода ЭВМ.

\subsubsection{} 
Все  части  пояснительной  записки  необходимо  излагать  только  на одном языке --- на русском или белорусском, или на одном из иностранных языков, например английском или немецком.

\subsubsection{} 
Описки и графические неточности, обнаруженные в тексте пояснительной записки, выполненной рукописным способом, допускается исправлять подчисткой, закрашиванием белой краской и нанесением на том же месте исправленного текста. Помарки и следы не полностью удаленного прежнего текста не допускаются.

\subsubsection{} 
Пояснительная записка\footnote{Пример сноски} должна быть оформлена в жестком переплете (в специальной папке для дипломных проектов или работ).


\subsection{Рубрикации, заголовки и содержание}

\subsubsection{} 
Текст пояснительной записки разделяют на логически сопряженные части --- разделы, а при необходимости и подразделы. Как разделы, так и подразделы могут состоять из одного или нескольких пунктов.

\subsubsection{}
Разделы должны иметь порядковые  номера,  обозначаемые арабски-ми цифрами без точки  в конце  и записанные с абзацного отступа. Подразделы 
нумеруют в пределах раздела, к которому они относятся.

\subsubsection{}
Иногда внутри подраздела необходимо выделить более мелкие смысловые подразделения – пункты, например: характеристики устройств и функциональных элементов технической системы; обоснование этапов планируемого эксперимента, характеристики аппаратов и приборов, необходимых для испытаний; показатели качества технической системы в различных режимах ее работы и т.\,д. В подобных случаях пункты нумеруют в пределах подраздела. Цифровой индекс пункта должен состоять из номеров раздела, подраздела и пункта, разделенных точками, и записан с абзацного отступа. 

Пункты при необходимости могут быть разбиты на подпункты, которые нумеруются в пределах каждого пункта. 

\subsubsection{}
Если в пояснительной записке выделены только разделы, то пункты нумеруют в пределах раздела.

\subsubsection{}
Каждый раздел и подраздел должен иметь краткий и ясный заголо-вок. Пункты, как правило, заголовков не имеют. Заголовки разделов записывают прописными буквами без точки в конце заголовка. Заголовки подразделов записывают стро чными буквами, начиная с первой прописной. Заголовки не подчеркивают. Перен осы слов в заголовках не допускаются. Если заголовок состоит из двух предложений, их разделяют точкой.

В случае, когда заголовки раздела или подраздела занимают несколько строк, то строки выравниваются  по первой букве  заголовка  в соответствии с приложением Л\footnote{Его тут нет}.

\subsubsection{}
Каждый раздел пояснительной записки рекомендуется начинать с новой страницы. 

Между заголовком раздела (подраздела) и текстом оставляют пробельную строку --- при компьютерном способе выполнения  записки;  интервал  шириной 15\,мм --- при рукописном способе (см.~приложение Л).

Между заголовками разделов и входящих в него подразделов допускается помещать небольшой вводный текст, предваряющий подраздел.

\subsubsection{}
Перечень всех разделов и подразделов, включающий порядковые номера и заголовки, оформляют в виде содержания --- обязательного элемента пояснительной записки. Содержание помещают непосредственно за заданием на проектирование и включают в общую нумерацию страниц.

Слово \MakeUppercase{содержание} записывают прописными буквами полужирным шрифтом 14~---~16 пунктов и располагают по центру строки. Между словом \MakeUppercase{содержание} и самим содержанием оставляют промежуток, равный пробельной строке. В содержании заголовки выравнивают, соподчиняя по разделам, подразделам и пунктам (если последние имеют заголовки), смещая вертикали вправо относительно друг друга на 2 знака.

% Пример организации пустых разделов
\section*{Введение}
\label{sec:intro}
\addcontentsline{toc}{section}{Введение} \newpage

\section{Анализ нескорректированной системы управления} \newpage
\label{sec:analysys_equations}

\subsection{Анализ исходных данных} \newpage 
\label{sec:analysys_data}

\subsection{Статические и динамические характеристики элементов системы} \newpage 
\label{sec:stat_and_dyn}

\subsection{Структурная схема нескорректированной системы} \newpage 
\label{sec:str_schema}

\subsection{Определение желаемого коэффициента усиления разомкнутой системы} \newpage 
\label{sec:determ_factor}

\subsection{Анализ устойчивости} \newpage 
\label{sec:analysys_rob}

\subsection{Выводы} \newpage 
\label{sec:conclusion}

\subsubsection{Пункт подраздела, не должен появится в содержании} \newpage

\section{Синтез корректирующих устройств} \newpage

\section*{Приложение А (информационное)  Пример заполнения титульного листа} 
\addcontentsline{toc}{section}{Приложение А (информационное)  Пример заполнения титульного листа} \newpage

\cite{Morozov_2011}

\cite[книженция]{kulezin_2004}

\cite[книженция]{kulezin_2004}

\cite{guk_1999}

\cite{kluev_1989}

\cite{cite_webpage}

\cite{microproc_1988}
